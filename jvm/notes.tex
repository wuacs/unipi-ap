\documentclass[a4paper,10pt]{article}
\newcommand{\myparagraph}[1]{\paragraph{#1}\mbox{}\\}
\usepackage[a4paper,margin=3.5cm]{geometry} %Sets the page geometry
\usepackage{url}
\usepackage{dirtytalk}
\usepackage{graphicx} % Package for \includegraphics
\usepackage{wrapfig} % Figure wrapping
\usepackage[T1]{fontenc} % Output font encoding for international characters
\setlength{\parskip}{1em} % Set space when paragraphs are used
\usepackage{amssymb}
\usepackage{amsmath}
\usepackage{tcolorbox}
\usepackage{mathtools}

% Lets you use \blankpage to make a blank page
\newcommand{\blankpage}{
\newpage
\thispagestyle{empty}
\mbox{}
\newpage
}

\usepackage{listings, xcolor}
\lstset{
tabsize = 4, %% set tab space width
showstringspaces = false, %% prevent space marking in strings, string is defined as the text that is generally printed directly to the console
numbers = left, %% display line numbers on the left
commentstyle = \color{green}, %% set comment color
keywordstyle = \color{blue}, %% set keyword color
stringstyle = \color{red}, %% set string color
rulecolor = \color{black}, %% set frame color to avoid being affected by text color
basicstyle = \small \ttfamily , %% set listing font and size
breaklines = true, %% enable line breaking
numberstyle = \tiny,
}

% Self Explanatory
\newtheorem{theorem}{Theorem}[section]
\newtheorem{definition}{Definition}[section]
\newtheorem{corollary}{Corollary}[theorem]
\newtheorem{lemma}[theorem]{Lemma}

\newcommand{\dollar}{\mbox{\textdollar}}

% Other
\DeclarePairedDelimiter\floor{\lfloor}{\rfloor} %Floor function

\begin{document}

\section{JVM's General Architecture}

Main subjects:

\begin{itemize}
    \item Runtime Systems
    \item The Java RunTime Environment (JRE)
    \item JVM as an abstract machine
    \item JVM Data Types
    \item JVM RunTime Data Areas
    \begin{itemize}
        \item Thread
        \item Frames
        \item Heap
        \item Non - Heap
        \item Class Data
    \end{itemize}
    \item Method Area
    \item Disassembling Java Class Files (Javap)
    \item Constant Pool
    \item Loading, linking and veryfing class files
    \item Initialization
    \item JVM exit
\end{itemize}

Notes:

\begin{itemize}

    \item JVM is a \textbf{multi-thread stack based} machine, which means that the memory model supposed by the JVM is a \underline{stack} and it is \textbf{multi-thread} because it allows for parallel execution model(using things like \verb|Threads|, \verb|Runnables|)

    \item On slide 14 it is said that there is \textbf{no type information on local variables at runtime}: 
     \begin{itemize}
        \item This is only true for a $\verb|class|$ file of version previous than 50.0
        \item New versions of the JVM attach to every method a $\verb|StackMapTable Attribute|$ which should map type information for local variables.
     \end{itemize}
     
     \item Interestingly, each Thread has 3 areas: its \textbf{p}rogram \textbf{c}ounter, the \textbf{java stack} and the \textbf{native stack}:
     \begin{itemize}
        \item \textbf{Native stack} contains the information about native functions(in C/C++) called. It is non-blocking the call to the native code so I dont know which message mechanism is implemented for the callback.
     \end{itemize} 
     
     \item Interestingly, at least for Hotspot JVM, a java \verb|Thread| is mapped to a \textit{native} Thread(i.e. a thread in the operating system running the jvm). So when a Java Thread has to be started, first everything needed is initialized(pc, stack ecc.), then we need the native thread to call the \verb|run()| method on the java's \verb|Thread| object. The \verb|run| call is blocking, thus the OS's thread will only be released when the Java Thread returns.
          
\end{itemize}

\section{JVM's instruction architecture}

\begin{itemize}
    \item The operand stack has 32 bit pushable words.
        \begin{itemize}
            \item Nevertheless, instructions are of \textit{variable} length, meaning an instruction is made of two parts, where the second is of variable length:
            \begin{itemize}
                \item The \textit{opcode}: one byte representing the operation
                \item The arguments: one or more byte of arguments.
            \end{itemize}
        \end{itemize}
     \item The are \textbf{4} ways we can address arguments:
     \begin{itemize}
         \item \textbf{Immediate Addressing}: the argument is embedded in the opcode: \verb|iload_1| pushes integer 1 on the stack. Its hex repres. is: \verb|00 00 00 1b|.
         \item \textbf{Variable Addressing}: the \verb|load| and \verb|store| access variables from the \textit{local variable array} by, respectively, pushing something from the array onto the stack and by popping something from the stack and inserting it in the locals area. 
        \item \textbf{Stack Addressing}: the arguments are taken from the top of the stack, which implies most of the time, popping it. E.g. \verb|ireturn| pops an integer and returns it.
        \item \textbf{Literal Addressing}: refers to one entry in the \textit{constant pool}, similiar to \verb|Variable Addressing| but for constant pool. E.g. \verb|ldc| (load constant) pushes a constant \#index from a constant pool (\verb|String|, \verb|int|, \verb|float|, \verb|Class|,\\ \verb|java.lang.invoke.MethodType|, \verb|java.lang.invoke.MethodHandle|, or a\\ \verb|dynamically-computed constant)| onto the stack. 
     \end{itemize}
     \item Type of method invocation:
     \begin{itemize}
         \item \textbf{invoke\_virtual} (virtual is another word for dynamic dispatch): 
         \begin{lstlisting}[language = Java , frame = trBL , firstnumber = last , escapeinside={(*@}{@*)}] 
int add12and13() {
    return addTwo(12, 13);
}
         \end{lstlisting}
         becomes:
         \begin{lstlisting}[mathescape]
         int add12and13();
         descriptor: ()I
         flags: (0x0000)
         Code:
            stack=3, locals=1, args_size=1
                0: aload_0
                1: bipush        12
                3: bipush        13
                5: invokevirtual #7   // Method addTwo:(II)I
                8: ireturn
            LineNumberTable:
                line 8: 0
\end{lstlisting}
    \item \textbf{invoke\_special} (special means we know the class of object on which the method is called at compile time)
    \begin{description}
        \item No static flow analysis is done! Thus only really explicit calls are counted e.g. static methods.
        \item \begin{lstlisting}[language = Java , frame = trBL , firstnumber = last , escapeinside={(*@}{@*)}] 
class Invoke {
    static int add12and13() {
        return 12+13;
    }
    
    public static void main(String[] args) {
        Invoke.add12and13();
        
    }
}
            \end{lstlisting}
            becomes
       \item
      \begin{lstlisting}
    public static void main(java.lang.String[]);
           descriptor: ([Ljava/lang/String;)V
            flags: (0x0009) ACC_PUBLIC, ACC_STATIC
            Code:
                stack=1, locals=1, args_size=1
                    0: invokestatic  #7  // add12and13:()I
                    3: pop
                    4: return
      \end{lstlisting}
    \end{description}
     \end{itemize}
\end{itemize}

\end{document}